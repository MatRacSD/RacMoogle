\documentclass{beamer}

\usepackage[utf8]{inputenc}
\usepackage[spanish]{babel}
\usepackage{graphicx}

\title{Moogle!}
\author{Raciel Alejandro Simón Domenech}
\institute{MATCOM}
\date{}

\begin{document}

\begin{frame}
\titlepage
\end{frame}

\begin{frame}{Descripción}
Moogle! es una aplicación \textit{totalmente original} cuyo propósito es buscar inteligentemente un texto en un conjunto de documentos.

\bigskip

Es una aplicación web, desarrollada con tecnología .NET Core 6.0, específicamente usando Blazor como \textit{framework} web para la interfaz gráfica, y en el lenguaje Csharp.

\bigskip

La aplicación está dividida en dos componentes fundamentales:
\begin{itemize}
\item MoogleServer: es un servidor web que renderiza la interfaz gráfica y sirve los resultados.
\item MoogleEngine: es una biblioteca de clases donde está implementada la lógica del algoritmo de búsqueda.
\end{itemize}
\end{frame}

\begin{frame}{Funcionamiento}
La búsqueda se realiza utilizando el modelo vectorial del TF-IDF, para lo cual se expresa cada documento y el query como vectores, entonces se calcula el coseno de similitud de cada vector documento con el vector query y se establece este valor como score, valor que determina la importancia del documento.

\bigskip

Al iniciar el programa, se indexan todos los documentos con extensión .txt de la carpeta Content, en caso de no existir dicha carpeta, se lanzará una excepción.

\bigskip

Después se construye un vector donde cada posición la ocupa cada palabra que exista en el documento, excluyendo las palabras que carecen de relevancia como conjunciones, preposiciones, etc.

\bigskip

Posteriormente, se vectoriza cada documento, expresándolos como un vector donde cada posición contiene el peso TF-IDF de una palabra de la colección de documentos en dicho documento.

\bigskip

Una vez completada dicha operación se procede a extraer las raíces de las palabras del vector utilizando la librería SharpNL y se calculan los pesos TF-IDF de dichas raíces.
\end{frame}

\begin{frame}{Funcionamiento}
Una vez realizado el preprocesamiento de los documentos, el programa termina de iniciar y espera por el input del usuario.

\bigskip

Una vez el usuario ha introducido la consulta, esta se vectoriza, se calculan los pesos tf-idf de sus palabras, posteriormente se busca en la colección de documentos, utilizando la distancia de Damerau - Levenshtein, aquellas palabras de la consulta que no aparecen en ningún documento.

\bigskip

Si se encuentran palabras similares, se sustituyen por las que no aparecieron y se calcula sus pesos tf-idf.

\bigskip

Después se determina el coseno de similitud entre el vector de la consulta y los vectores de los documentos, y el valor resultante se le asigna como score (relevancia del documento según la consulta del usuario) a cada documento.

\bigskip

Una vez concluido esto, se llevan las palabras de la consulta a sus raíces, se vectoriza, se calculan sus pesos tf-idf y el coseno de similitud entre los vectores de las palabras raíces de cada documento y la de la consulta, y el valor obtenido se le suma a los scores obtenidos.

\bigskip

Se ordenan y se devuelven en orden descendente los cinco primeros documentos cuyo score sea mayor que 0.

\bigskip

En caso de no existir ningún documento con score mayor que 0 se devuelve un mensaje que indica que no hay resultados para la consulta realizada.
\end{frame}

\begin{frame}{Complementos}

El programa cuenta con varios complementos que pueden resultar útiles para realizar búsquedas más específicas u ofrecer información adicional al usuario:

\begin{itemize}
\item Operadores para búsquedas más específicas (!, \^{}, *, ~)
\item Sugerencia de términos relacionados
\item Snippet de texto relevante en los resultados
\end{itemize}

\end{frame}



\end{document}